\documentclass[10pt,a4paper]{article}
\linespread{1.1}
% layout/typography
\usepackage[hmargin=2.75cm,vmargin=3cm]{geometry}
\usepackage[english]{babel} % use Dutch hyphenation etc.
\usepackage{parskip}         % don't indent paragraphs and leave vertical space instead
\usepackage[T1]{fontenc}  % make accented work properly in pdf
\renewcommand{\thefootnote}{\fnsymbol{footnote}} % use symbols for footnotes

% math stuff
\usepackage{amssymb,amsmath,amsthm,dsfont,hyperref}

% set up alternative commands to allow for easy adaptation to the notation of the textbook.
\newcommand{\sub}{\subset}           % new command to refer to standard notation \subset for standard subset (which might be equal)
\newcommand{\subneq}{\subsetneq} % new command to refer to standard notation \subsetneq for proper subset (not allowed to be equal)
% Uncomment the lines below to use the notation of the textbook
%    Comment the lines below to use more common notations (for (proper) subsets, negation and set difference)
\renewcommand{\sub}{\subseteq}      % use \subseteq instead of \subset for standard subset (which might be equal)
%\renewcommand{\subneq}{\subset} % use \subset instead of \subsetneq for proper subset (which is not allowed to be equal)
%\renewcommand{\setminus}{-}        % use - instead of \ for relative complement of sets
%\renewcommand{\neg}{{\sim}}       % use ~ as negation symbol

% misc.
\usepackage{enumitem} % for the [resume] option
\renewcommand{\theenumi}{\alph{enumi}}
\renewcommand{\theenumii}{\roman{enumii}}

\usepackage{graphicx,caption}
\usepackage{threeparttable}
\captionsetup{labelfont = sc, textfont = it}

\makeatletter
\renewcommand*\env@matrix[1][*\c@MaxMatrixCols c]{%
	\hskip -\arraycolsep
	\let\@ifnextchar\new@ifnextchar
	\array{#1}}

\newcommand{\norm}[1]{\lvert\lvert #1 \rvert\rvert}

\usepackage{tikz,xcolor}
\usepackage{ upgreek }


\def\point[#1]{
	\draw (#1) node[fill=black,thick,circle,minimum size=2.2pt,inner sep=0pt,outer sep=0pt] {};
	\draw (#1) node[fill=white,semithick,circle,minimum size=1pt,inner sep=0pt,outer sep=0pt] {};
}

\newcommand{\R}{\mathbb{R}}
\newcommand{\x}{\textbf{x}}
\newcommand{\y}{\textbf{y}}
\newcommand{\0}{\textbf{0}}
\renewcommand{\l}{\langle}
\renewcommand{\r}{\rangle}
\renewcommand{\v}{\textbf{v}}
\newcommand{\z}{\textbf{z}}
\newcommand{\N}{\mathbb{N}}
\newcommand{\Q}{\mathbb{Q}}
\newcommand{\C}{\mathbb{C}}
\newcommand{\Z}{\mathbb{Z}}
\renewcommand{\P}{\mathbb{P}}
\renewcommand{\O}{\mathcal{O}}
\newcommand{\F}{\mathcal{F}}
\newcommand{\E}{\mathbb{E}}
\newcommand{\A}{\mathbb{A}}
\renewcommand{\a}{\alpha}
\renewcommand{\b}{\beta}
\newcommand{\e}{\varepsilon}
\newcommand{\p}{\varphi}
\renewcommand{\d}[2]{\frac{d#1}{d#2}}
\newtheorem{theorem}{Theorem}
\newtheorem{lemma}[theorem]{Lemma}
\newcommand{\aaa}{(a_1:\cdots:a_r)}
\newcommand{\bbb}{(b_1:\cdots:b_s)}
\newcommand{\ab}{(a_1b_1:a_1b_2:\cdots:a_rb_s)}
\renewcommand{\Pr}{\P^{r-1}}
\newcommand{\Ps}{\P^{s-1}}



\def\subto{\mathrel{\vcenter{\hbox{\scalebox{.6}{$\supset$}}}\kern-.4em\to}}
\usepackage{lmodern} 

\usepackage{color}
\definecolor{keywordcolor}{rgb}{0.7, 0.1, 0.1}   % red
\definecolor{commentcolor}{rgb}{0.4, 0.4, 0.4}   % grey
\definecolor{symbolcolor}{rgb}{0.0, 0.1, 0.6}    % blue
\definecolor{sortcolor}{rgb}{0.1, 0.5, 0.1}      % green
\definecolor{errorcolor}{rgb}{1, 0, 0}           % bright red
\definecolor{stringcolor}{rgb}{0.5, 0.3, 0.2}    % brown

%\usepackage{listings}
%\def\lstlanguagefiles{lstlean.tex}
%\lstset{language=lean}


\begin{document}
	\section*{\vspace{-1.5cm}\begin{center}\normalsize\normalfont{Homework 6, Formal Methods in Mathematics\\University Utrecht, \today\\AKS primality group\\Lena van Dongen, Roxy van de Kuilen, Tamira Lopes, Francois van der Rhee}\end{center}}
	\vspace{-\baselineskip}\noindent \rule{\textwidth}{0.025cm}\vspace{\parskip}
	For the final project, we would like to explore the AKS primality test in Lean. This means that we will be occupied with the number theory associated to this problem. 
	In a later version of the original proof, we see that this entails inequalities, as well as modulo calculations and cyclic groups. In the final parts of the proof, we will need to use cyclotomic polynomials.
	Furthermore, we will need to prove some results about the cardinality of some group generated by $X$, $X+1$, ..., $X+l$ in the field $F_P[X]/(h(X))$. 
	If we have enough time left, we also want to look into proving the time complexity of this algorithm.

	
	

	Define algorithm in Lean
	Prove the iff statement (Theorem 4.1 in PRIMES in P), using the theorems, lemmas and definitions as in PRIMES in P.\\
	Need to do:
	\begin{itemize}
		\item Search for useful sources
		\item definition algorithm
		\item Lemma 2.1
		\item Theorem 4.1
		\item Lemma 4.2
		\item Lemma 4.3
		\item Definition 4.4
		\item Lemma 4.5
		\item Lemma 4.6
		\item Lemma 4.7
		\item Lemma 4.8
		\item Lemma 4.9
	\end{itemize}
	
	Together: define algorithm + AKS theorem statement
	Individual labour??
	
	First Lean formalizations???
\end{document}